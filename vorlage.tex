%%%%%%%%%%%%%%%%%%% vorlage.tex %%%%%%%%%%%%%%%%%%%%%%%%%%%%%
%
% Beispiel-Vorlage zur Erstellung von Projekt-Dokumentationen
%
% Benutzen Sie bitte diese Datei um Ihre Dokumente zu erstellen
%
%%%%%% erstellt anhand svmono-Springer-Verlag-Vorlage %%%%%%%%%


%%%%%%%%%%%%%%%%%%%%%%%%%%%%%%%%%%%%%%%%%%%%%%%%%%%
\documentclass[envcountsame,envcountchap,deutsch]{i-studis}
\usepackage{makeidx}   % Erlaubt die Erzeugung eines Index-Verzeichnisses
\usepackage{multicol}  % Zweispaltiger Index-Verzeichnis
%\usepackage[bottom]{footmisc} % Erzeugung von Fußnoten nur beim Bedarf einbinden

%%--------------------------------------------------------
\usepackage[pdftex]{graphicx}
\usepackage[pdftex,plainpages=false]{hyperref}
%%-----------------------------------------------------
\usepackage{color}	    % Farbverwaltung
\usepackage{ngerman}     % Neue deutsche Rechtsschreibung
\usepackage[T1]{fontenc} % Silbentrennung 
\usepackage[english, ngerman]{babel}
% Alle eingebundenen Dateien sollten die passenden Encodings haben
% \usepackage[ansinew]{inputenc}  % Umlaute für Windows (öäüß)
% \usepackage[latin1]{inputenc}   % Umlaute für Linux
% \usepackage[applemac]{inputenc} % Umlaute für Mac
\usepackage[utf8]{inputenc}     % generelle Umlaute-Darstellung
%-----------------------------------------------------
\usepackage{listings} % Optionen für Code-Darstellung
\lstset
{% general command to set parameter(s)
	basicstyle=\scriptsize, % print whole listing small
	keywordstyle=\color{blue}\bfseries,
	% underlined bold black keywords
	identifierstyle=, % nothing happens
	commentstyle=\color{white}, % white comments
	stringstyle=\ttfamily, % typewriter type for strings
	showstringspaces=false, % no special string spaces
	framexleftmargin=7mm, 
	tabsize=3,
	showtabs=false,
	frame=single, 
	rulesepcolor=\color{blue},
	numbers=left,
	linewidth=146mm,
	xleftmargin=8mm
}
\usepackage{textcomp} % celsius - Darstellung
\usepackage{
	amssymb,
	amsfonts,
	amstext,
	amsmath
} % Mathematische Symbole
\usepackage[german, ruled, vlined]{algorithm2e}
\usepackage[a4paper]{geometry} %Andere Formatierung
\usepackage{bibgerm}
\usepackage{array}
\hyphenation{Test-ein-gabe} %Silbentrennung bei falschen Trennung anfügen
\setlength{\textheight}{1.1\textheight}
\pagestyle{myheadings} % Erzeugt selbstdefinierte Kopfzeile
\makeindex % Index-Erstellung

% Ab hier beginnt das eigentliche Dokument.
%--------------------------------------------------------------------------
\begin{document}

%------------------------- Titelblatt -------------------------------------
\title{Titel der Projekt- Abschlussarbeit auf Deutsch}
\subtitle{English Title of Project Work}
%---- Die Art der Dokumentation kann hier ausgewählt werden---------------
%\project{Master Abschlussarbeit}
%\project{Master Projektstudium}
%\project{Bachelor Abschlussarbeit}
\project{Projektarbeit}
%\project{Seminar zur Vorlesung ...}
% \project{Hausarbeit zur Vorlesung ...}
%--------------------------------------------------------------------------
\supervisor{Titel Vorname Name} % Betreuer der Arbeit
\author{Autor}                  % Autor der Arbeit
\address{Ort, den}              % hier wird der Ort eingetragen
\submitdate{Abgabedatum}        % Abgabedatum \today :)

\begingroup
  \renewcommand{\thepage}{Titel}
  \mytitlepage
  \newpage
\endgroup
%--------------------------------------------------------------------------
\frontmatter 
%--------------------------------------------------------------------------
\input{chapters/Danksagungen} %Danksagungen
\input{chapters/Vorwort}
\kurzfassung
Hier soll eine Kurzfassung der Arbeit stehen. Eine halbe Seite sollte genügen. Die Kurzfassung gibt in einer verständlichen Form den Gegenstand und das Ergebnis der Arbeit an. Sie soll dem Leser vermitteln, um was es geht und was die Leistung der Arbeit ist. Damit kann der Leser entscheiden, ob Thema, Inhalt und Ergebnis der Arbeit für ihn so interessant sind, dass er sie liest.
\\[3ex]
\noindent
The same in english (optional).

 % Kurzfassung Deutsch/English
\tableofcontents %Inhaltsverzeichnis - notwendig
% \listoffigures % Abbildungsverzeichnis - optional
% \listoftables % Tabellenverzeichnis- optional
%--------------------------------------------------------------------------
\mainmatter                     %Hauptteil (ab hier in arab. Seitenzahlen)
%--------------------------------------------------------------------------
% Kapitell werden einzeln abgespeichert und hier eingefügt
\chapter{Einleitung}

Begonnen werden soll mit einer Einleitung zum Thema: z.B. Hintergrund und Ziel (was, warum).

\chapter{Problemstellung}

Hier wird i.d.R. zunächst das generell vorliegende Problem diskutiert: Was ist zu lösen - was gibt es bisher an Lösungsansätzen (prinzipiell) und warum ist es wichtig, dass man dieses Problem löst. Letzteres ergibt sich oftmals aus der vorliegenden Anwendungssituation: Man braucht die Lösung, um eine bestimmte Aufgabe zu erledigen, ein System aufzubauen etc. Der Bezug auf vorhandene oder auch bisher fehlende Lösungen begründet auch die Intension und Bedeutung dieser Arbeit. Dies können allgemeine Gesichtspunkte sein - man liefert einen Beitrag für ein generell erkanntes oder zu erkennendes Problem - oder man hat eben eine spezielle Systemumgebung oder Produkt (z.B. in einer Firma u.s.w.), woraus sich dieses noch zu lösende Problem ergibt.

Die genaue Problematik und Randbedingungen werden dann in Kapitel \hyperref[Aufgabenstellung]{Kapitel~\ref{Aufgabenstellung}}
dargestellt.

\chapter{Aufgabenstellung und Zielsetzung}\label{Aufgabenstellung}

Hier wird nun die Aufgabenstellung konkret dargestellt: Was ist spezifisch zu lösen? Welche Randbedingungen sind prinzipiell gegeben und was ist die Zielsetzung? Letztere soll das beschreiben, was man mit dieser Arbeit (mindestens) erreichen möchte.
\input{chapters/Inhalt}
\input{chapters/Stile}
\input{chapters/Beispiel}
% ...

%--------------------------------------------------------------------------
\backmatter                     %Anhang
%-------------------------------------------------------------------------
%Literaturverzeichnis - notwendig
\bibliographystyle{gerplain}
\bibliographystyle{geralpha}
\bibliography{literatur}     % BibTeX-File ist literatur.bib
%--------------------------------------------------------------------------
\printindex % Index ausdrucken - optional
%--------------------------------------------------------------------------
% Anhänge
\begin{appendix}
   \include{chapters/AnhangA} %optional  
   \include{chapters/AnhangB} %optional
\end{appendix}
\end{document}
